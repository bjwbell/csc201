% Example LaTeX document for GP111 - note % sign indicates a comment
\documentclass[11pt]{article}
\usepackage{amsmath}
\usepackage{amsfonts}
% Default margins are too wide all the way around. I reset them here
\setlength{\topmargin}{-.5in}
\setlength{\textheight}{9in}
\setlength{\oddsidemargin}{.125in}
\setlength{\textwidth}{6.25in}
\begin{document}
\title{CSC 201 Homework 1}
\author{Brian Bell\\
}
\renewcommand{\today}{February 8, 2009}
\maketitle

\section*{Problem 1}
\begin{enumerate}
\item square: $\mathbb{R} \rightarrow \mathbb{R}$.
\item absolute : $\mathbb{R} \rightarrow \mathbb{R}$.
\item ``+'': $\mathbb{R} \times \mathbb{R} \rightarrow \mathbb{R}$.
\item ``-'': $\mathbb{R} \times \mathbb{R} \rightarrow \mathbb{R}$.
\item ``*'': $\mathbb{R} \times \mathbb{R} \rightarrow \mathbb{R}$.
\item ``/'': $\mathbb{R} \times \mathbb{R'} \rightarrow \mathbb{R}$.
  
\end{enumerate}

\section*{Problem 2}
\begin{enumerate}
\item The signature of iterate is $(f:A \rightarrow A) \times \mathbb{N} \rightarrow A.$
\item The signature of curry(iterate) is $(f: A \rightarrow A) \rightarrow (\mathbb{N} \rightarrow A).$
\end{enumerate}
\section*{Problem 3}
\begin{description}
\item[(a)] Right associativity
\item[(b)] See attached hand-drawn tree
\item[(c)] Change expression and term to \newline \newline
\begin{center}
<expression> ::= <term> | <term> + <expression-minus>
<expression-minus> ::= <term> | <term> - <expression-minus>
<term> ::= <term-divide> | <term-divide> * <term>
<term-divide> ::= <factor> | <factor> * <term-divide>
\end{center}
\item[(d)] I enforced right associativity and also ``-'' has higher precedence than ``+'' and ``/'' has higher precendence than ``*''.
\item[(e)] Yes the associativity is the correct one as commonly understood for mathematical expressions.
\end{description}
\end{document}
